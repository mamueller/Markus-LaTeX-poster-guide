
% vim:set ff=unix expandtab ts=2 sw=2:
%\begin{columns}
%  \newlength{\lc}
%  \setlength{\lc}{0.4\textwidth}
%  \newlength{\rc}
%  \setlength{\rc}{\dimexpr(\textwidth-\lc) \relax}
%  \begin{column}{\lc}
\begin{multicols}{2}
  The graphs show the reactions of a prototypical class of nonlinear two pool soil models to a disturbing time varying signal. 
  This model is a technically simple place holder for ecologically motivated nonlinear systems like the soil models mentioned above to be analyzed in the future. It is given by:\\
  \begin{eqnarray*}
  \dot{C}_x=I_{x}(t)  - \left(C_{x}^{2} + C_{x}\right) \operatorname{k_{x}}{\left (t \right )}\\
  \dot{C}_y=I_{y}(t)  - \left(C_{y}^{2} + C_{y}\right) \operatorname{k_{y}}{\left (t \right )}
  \end{eqnarray*}
  where $C_x,C_y$ are the carbon contents of two unconnected pools and the bounded  periodic functions $k_x(t) $ and $k_y(t) $ with:\\ 
  \begin{eqnarray*}
  k_{x_{min}}\le  k_x(t) \le k_{x_{max}} \\k_{y_{min}} \le  k_y(t) \le k_{y_{max}}
  \end{eqnarray*}
  describe the seasonal changes in decomposition speed.
  e.g.:\\ 
  \begin{eqnarray*}
  k_x=\frac{k_{xmax}}{2} + \frac{k_{xmin}}{2} + \frac{1}{2} \left(k_{xmax} - k_{xmin}\right) \sin{\left (4 t \right )}\\k_y=\frac{k_{ymax}}{2} + \frac{k_{ymin}}{2} + \frac{1}{2} \left(k_{ymax} - k_{ymin}\right) \sin{\left (4 t \right )}
  \end{eqnarray*}
  The system can have a fixed point: 
  $$ \mathbf{C}_f= \begin{pmatrix} C_{fx} \\ C_{fy} \end{pmatrix} $$
  if the input streams have the same period and phase as the decomposition rates. For constant input streams it stays in a predictable region (an invariant set in the phase plane) \\ 
  \begin{eqnarray*}
  \mathbf{I}_0(t)=\left(
      \begin{pmatrix}
        \left( C_{fx}^{2} + C_{fx} \right) \operatorname{k_{x}}{\left (t \right )}\\
        \left( C_{fy}^{2} + C_{fy} \right)  \operatorname{k_{y}}{\left (t \right )}
      \end{pmatrix}
    \right)
  \end{eqnarray*}
  The fixpoint would be.
  \[
  \mathcal{A}=\{\mathbf{C}_f\}
  \]
  We will now disturb both mass influxes individually by perturbations $u_x(t),u_y(t)$ and get: 
  \begin{eqnarray*}
  \dot{C}_x=I_{0 x}(t) + u_{x}(t) - \left(C_{x}^{2} + C_{x}\right) \operatorname{k_{x}}{\left (t \right )}\\
  \dot{C}_y=I_{0 y}(t) + u_{y}(t) - \left(C_{y}^{2} + C_{y}\right) \operatorname{k_{y}}{\left (t \right )}
  \end{eqnarray*}
  The plots show the typical behavior of an ISS system: The changes in the state variables will asymptotically converge to a region of stability around an invariant set, whose size is a monotone function of the size of the disturbance (denoted by $|u|_{\infty}$).
  For this particular system we proved the ISS property rigorously. The proof relies on the construction of an ISS Lyapunov function whose choice is \emph{not determined but  inspired} by a property of the system interpretable in ecologically terms. Expressed casually: "The  system can counteract supply changes fast enough".
  This situation seems to be typical: The problem of establishing ISS for e.g. all the $\mathbf{I},\mathbf{T},\mathbf{N}$ models based on the ecologic principles they follow , is too hard.
  But bio-chemical of biophysical restrictions could provide clues to ISS proofs for a particular system.
  
%  \end{column}
%  \begin{column}{\rc}
    \includegraphics[width=\columnwidth,clip=true,trim=1.0cm 0cm 2cm 1cm]{combiPlot2.pdf}
    \begin{enumerate}
    	\item
    	The four plots on the top show the disturbances.
    	\item
    	The next four plots in the middle show the effect of this disturbances on the solutions for a system with fixed point. 
    	\item
    	The next four plots in the middle show the effect of this disturbances on the solutions for the system which no longer has a fixed point, but at least an invariant set, the dark blue square in the middle.
    \end{enumerate}
 % \end{column}
%\end{columns}
\end{multicols}
\vspace{1cm}
