
% vim:set ff=unix expandtab ts=2 sw=2:

 
\alert{\textit{Vertical alignment is simpler than with beamer Boxes but still challenging:}}
\\
Since the textblocks as Beamer boxes  adapt their vertical extent to their content.
The vertical extent can only be controlled indirectly by putting minipages inside.
This will also only work if the text does not fill the box completely.
Fortunately the textblocks have no margins (in the default setting) so that the
size of the inner box is identical to the outer size of the included minipage.


%A 
%\begin{verbatim}
%\newsavebox\mybox
%\setbox0=\vtop{%
%		\begin{block}{A standard block} 
%			some text  
%		\end{block}%
%}
%\newlength{\bh}
%\setlength{\bh}{\ht0}%
%\addtolength{\bh}{\dp0}%
%\end{verbatim}
%and computing the necessarry vspace adjustment for the last box with code like the following:
%{\small
%  \begin{verbatim}
%  \documentclass{beamer}
%  \newlength\A
%  \newlength\B
%  \newlength\C
%  \setlength\A{100pt}
%  \setlength\B{5pt}
%  \newcommand{\n}{3}
%  \newcommand{\m}{\numexpr \n -1  \relax}
%  
%  \setlength\C{\dimexpr (\A - \B * \n) / (\n-1)  }
%  \begin{document}
%  \the\m
%  \\
%  \the\C
%  \end{document}
%  \end{verbatim}
%}
%The resultign code is extremely messy and furthermore does not belong in the content of the document.
%\vspace{1cm}
%
%\alert{\textit{Conclusions}}
%\begin{itemize}
%   \item
%     If you need boxes aligned with top and bottom do not use simple beamer blocks.
%     \begin{verbatim}
%        \usepackage[most,poster]{tcolorbox}
%     \end{verbatim}
%     or you use our other template
%     \begin{verbatim}
%        baposterSAB2018.cls
%     \end{verbatim}
%     Both ideas are derived from baposter.cls
%     which implements vertical box alignment in the class away 
%     from usercode and even allows boxes to be positioned relative BETWEEN other
%     boxes. 
%     You still have to be carefull not to overload the boxes but everything else is much less painfull.
%\end{itemize}
%\vspace{1cm}
