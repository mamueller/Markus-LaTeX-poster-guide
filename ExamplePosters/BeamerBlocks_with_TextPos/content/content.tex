
% vim:set ff=unix expandtab ts=2 sw=2:


%\the\textheight
	\setlength{\TPHorizModule}{1\textwidth} % our coordinates are <1
	\setlength{\TPVertModule}{\TPHorizModule}
	\textblockorigin{50mm}{200mm}
	%start everything near the top-left corner
	\setlength{\parindent}{0pt} % 
	\definecolor{myblue}{RGB}{128,128,255}
	\definecolor{myred}{RGB}{255,128,128}

	\begin{textblock}{0.2}(0,0)
		This Block ck 3 modules wide, and is placed with its top left corner
		 ‘origin’ on the page. Note that the length of the block is not
		specified in the arguments -- the box will be as long as necessary to
		accomodate the text inside it.
		You need to examine the output of the
		text to adjust the positioning of the blocks on the page.
	\end{textblock}
	\begin{textblock}{.2}(.1,.2)
	  \textblocklabel{block two} 
    \begin{block}{Beamer Block inside}
      \begin{minipage}[t][50cm][t]{\textwidth}	
		    Here is another, slightly narrower, block, at position (.1,.2) on the page.
        That contains a beamer block environment, which contains a minipage to 
        have some (indirect) control over the size of the box.
        % vim:set ff=unix expandtab ts=2 sw=2:
        \alert{\textit{Example with pictures:}}
        \begin{itemize}
        	\item we add a pseudo picture
        \end{itemize}
        \vspace{1cm}
        \begin{columns}
        \column{.8\textwidth}
        	\begin{figure}[tb]
        	\begin{center}
        		\includegraphics[width=.55\textwidth]{images/content/FIG1.pdf}
        	\end{center}
        	\end{figure}
        \column{.2\textwidth}
        \small{\textit{a pseudo caption}}
        \vspace{4cm}
        	\begin{figure}[tb]
        		\includegraphics[width=.6\textwidth]{images/qrcode-RSCAPE.jpg}
        	\end{figure}
        \end{columns}
        \vspace{1cm}
        We add some text here to show that the block will grow even over the bottom of the column. This demonstrates
            \end{minipage}
           \end{block}
        \end{textblock}
        \begin{textblock}{.3}[0.5,0.5](.2,.9)
    {\textblockcolor{myblue}
		This Block is at position (2,3), but because the optional argument
		[0.5,0.5]
		has been given, it is the centre of the block which is
		located at that point, rather than the top-left corner.
    We also used 
    \begin{verbatim}
    \textblockcolor{myblue}
    \end{verbatim}
    }
	\end{textblock}
  % lets build our own environment with 
  % a minibox inside to be able to influence
  % the vertical 
  \newenvironment{mybox}[3]{
	  \begin{textblock*}{#1}#2
  	  \begin{minipage}[t][#3][t]{\textwidth}	
  }
  {
	    \end{minipage}
	  \end{textblock*}
  }
  %% now we set some lengths to organize our boxes in columns 
  %% for both columns
  \newlength{\vblocksep}
  \setlength{\vblocksep}{1cm} 
  \newlength{\hblocksep}
  \setlength{\hblocksep}{1cm} 
  \newlength{\xcols}
  \setlength{\xcols}{0.4\textwidth} 
  \newlength{\hcols}
  \setlength{\hcols}{80cm} %overall height of columns (fixme: this should ideally be set in the template)
  \newlength{\wcols}
  \setlength{\wcols}{40cm} %overall width of our columns
  %% column specific
  %% colA
  \newlength{\wcolA}
  % set the widht of the first column manually
  \setlength{\wcolA}{.41\wcols} 
  \newlength{\hboxOneA} 
  % set the height of the first box manually
  \setlength{\hboxOneA}{15cm}
  \newlength{\hboxTwoA}
  % compute the height of the second box 
  \setlength{\hboxTwoA}{ \dimexpr ( \hcols - \hboxOneA) \relax}
  
  %% now draw the actual boxes
  % columnA
  \begin{mybox}{\wcolA}{(\xcols,0cm)}{\hboxOneA}
      
% vim:set ff=unix expandtab ts=2 sw=2:

 
\alert{\textit{Vertical alignment is simpler than with beamer Boxes but still challenging:}}
\\
Since the textblocks as Beamer boxes  adapt their vertical extent to their content.
The vertical extent can only be controlled indirectly by putting minipages inside.
This will also only work if the text does not fill the box completely.
Fortunately the textblocks have no margins (in the default setting) so that the
size of the inner box is identical to the outer size of the included minipage.


%A 
%\begin{verbatim}
%\newsavebox\mybox
%\setbox0=\vtop{%
%		\begin{block}{A standard block} 
%			some text  
%		\end{block}%
%}
%\newlength{\bh}
%\setlength{\bh}{\ht0}%
%\addtolength{\bh}{\dp0}%
%\end{verbatim}
%and computing the necessarry vspace adjustment for the last box with code like the following:
%{\small
%  \begin{verbatim}
%  \documentclass{beamer}
%  \newlength\A
%  \newlength\B
%  \newlength\C
%  \setlength\A{100pt}
%  \setlength\B{5pt}
%  \newcommand{\n}{3}
%  \newcommand{\m}{\numexpr \n -1  \relax}
%  
%  \setlength\C{\dimexpr (\A - \B * \n) / (\n-1)  }
%  \begin{document}
%  \the\m
%  \\
%  \the\C
%  \end{document}
%  \end{verbatim}
%}
%The resultign code is extremely messy and furthermore does not belong in the content of the document.
%\vspace{1cm}
%
%\alert{\textit{Conclusions}}
%\begin{itemize}
%   \item
%     If you need boxes aligned with top and bottom do not use simple beamer blocks.
%     \begin{verbatim}
%        \usepackage[most,poster]{tcolorbox}
%     \end{verbatim}
%     or you use our other template
%     \begin{verbatim}
%        baposterSAB2018.cls
%     \end{verbatim}
%     Both ideas are derived from baposter.cls
%     which implements vertical box alignment in the class away 
%     from usercode and even allows boxes to be positioned relative BETWEEN other
%     boxes. 
%     You still have to be carefull not to overload the boxes but everything else is much less painfull.
%\end{itemize}
%\vspace{1cm}

  \end{mybox}
  \begin{mybox}{\wcolA}{(\xcols,\dimexpr \hboxOneA + \vblocksep \relax )}{\hboxTwoA}
    
% vim:set ff=unix expandtab ts=2 sw=2:
\alert{\textit{Example with pictures:}}
\begin{itemize}
	\item we add a pseudo picture
\end{itemize}
\vspace{1cm}
\begin{columns}
\column{.8\textwidth}
	\begin{figure}[tb]
	\begin{center}
		\includegraphics[width=.55\textwidth]{images/content/FIG1.pdf}
	\end{center}
	\end{figure}
\column{.2\textwidth}
\small{\textit{a pseudo caption}}
\vspace{4cm}
	\begin{figure}[tb]
		\includegraphics[width=.6\textwidth]{images/qrcode-RSCAPE.jpg}
	\end{figure}
\end{columns}
\vspace{1cm}
We add some text here to show that the block will grow even over the bottom of the column. This demonstrates

  \end{mybox}

  
  %% column specific
  %% colB
  \newlength{\wcolB}
  % compute the width of the second column 
  \setlength{\wcolB}{\dimexpr \wcols-\wcolA \relax} 
  \newlength{\hboxOneB}
  % columnB
  % Ttis time we even go a step further and measure
  % the contents of the first box before we draw it
  %\newsavebox\myMeasuringBox
  \setbox0=\vtop{%
  	\begin{minipage}[t]{\wcolB}	
      
% vim:set ff=unix expandtab ts=2 sw=2:
  {\vspace{.2cm}\textbf{REddyProc}\hfill\normalsize{her could be some names}}
\alert{\textit{Context:}}

To measure how much space this box will take we put it in a 
savebox:


	  \end{minipage}
  }
  \setlength{\hboxOneB}{\ht0}%
  \addtolength{\hboxOneB}{\dp0}%
  \newlength{\hboxTwoB}
  % compute the height of the second box 
  \setlength{\hboxTwoB}{ \dimexpr ( \hcols - \hboxOneB) \relax}
  
  \newlength{\xcolB}
  \setlength{\xcolB}{\dimexpr \xcols+\wcolA+\hblocksep \relax } 
  \begin{mybox}{\wcolB}{(\xcolB,0cm)}{\hboxOneB}
   \usebox0  
  \end{mybox}
  \begin{mybox}{\wcolB}{(\xcolB,\dimexpr \hboxOneB + \vblocksep \relax )}{\hboxTwoB}
    
% vim:set ff=unix expandtab ts=2 sw=2:
\alert{\textit{Example with pictures:}}
\begin{itemize}
	\item we add a pseudo picture
\end{itemize}
\vspace{1cm}
\begin{columns}
\column{.8\textwidth}
	\begin{figure}[tb]
	\begin{center}
		\includegraphics[width=.55\textwidth]{images/content/FIG1.pdf}
	\end{center}
	\end{figure}
\column{.2\textwidth}
\small{\textit{a pseudo caption}}
\vspace{4cm}
	\begin{figure}[tb]
		\includegraphics[width=.6\textwidth]{images/qrcode-RSCAPE.jpg}
	\end{figure}
\end{columns}
\vspace{1cm}
We add some text here to show that the block will grow even over the bottom of the column. This demonstrates

  \end{mybox}
