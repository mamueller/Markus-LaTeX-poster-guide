 
% vim:set ff=unix expandtab ts=2 sw=2:
%\documentclass[landscape,final,a0paper,fontscale=0.285]{baposter}
\documentclass[portrait,final,a0paper,fontscale=0.285]{baposter}

%\usepackage{grffile}
\usepackage{calc}
\usepackage{amsmath,amssymb,amsfonts,amscd,amsthm}
\usepackage{relsize}
\usepackage{multirow}
\usepackage{rotating}
\usepackage{url}
\usepackage{graphicx}
\usepackage{multicol}
\usepackage{longtable,booktabs}
\usepackage{array}
\usepackage{natbib}
\usepackage{bm}
\usepackage{bibentry}
%\usepackage[natbib=true]{biblatex}
%\usepackage{times}
%\usepackage{helvet}
%\usepackage{bookman}
%\usepackage{palatino}

\newcommand{\captionfont}{\footnotesize}
\newcommand{\SoilR}{\textsc{SoilR}}
\newcommand{\RC}{$\Delta^{14}$C}
\newcommand{\figsize}{0.108}

%\usetikzlibrary{calc}


%%%%%%%%%%%%%%%%%%%%%%%%%%%%%%%%%%%%%%%%%%%%%%%%%%%%%%%%%%%%%%%%%%%%%%%%%%%%%%%%
%%%% Some math symbols used in the text
%%%%%%%%%%%%%%%%%%%%%%%%%%%%%%%%%%%%%%%%%%%%%%%%%%%%%%%%%%%%%%%%%%%%%%%%%%%%%%%%

%%%%%%%%%%%%%%%%%%%%%%%%%%%%%%%%%%%%%%%%%%%%%%%%%%%%%%%%%%%%%%%%%%%%%%%%%%%%%%%%
% Multicol Settings
%%%%%%%%%%%%%%%%%%%%%%%%%%%%%%%%%%%%%%%%%%%%%%%%%%%%%%%%%%%%%%%%%%%%%%%%%%%%%%%%
\setlength{\columnsep}{1.5em}
\setlength{\columnseprule}{0mm}

%%%%%%%%%%%%%%%%%%%%%%%%%%%%%%%%%%%%%%%%%%%%%%%%%%%%%%%%%%%%%%%%%%%%%%%%%%%%%%%%
% Save space in lists. Use this after the opening of the list
%%%%%%%%%%%%%%%%%%%%%%%%%%%%%%%%%%%%%%%%%%%%%%%%%%%%%%%%%%%%%%%%%%%%%%%%%%%%%%%%
\newcommand{\compresslist}{%
\setlength{\itemsep}{1pt}%
\setlength{\parskip}{0pt}%
\setlength{\parsep}{0pt}%
}


\newcommand{\captionfont}{\footnotesize}
\newcommand{\SoilR}{\textsc{SoilR}}
\newcommand{\RC}{$\Delta^{14}$C}
\newcommand{\figsize}{0.108}
\DeclareRobustCommand{\diag3}[3]{
	\left(
		\begin{matrix}
		#1	& 0  		& \cdots	 & 0 \\
		0 	& \ddots	& \ddots 	 & \vdots \\
		\vdots 	& \ddots 	& \ddots 	 & 0 \\
		0 	& \cdots        & 0		 & #3\\
		\end{matrix}
	\right) 
}
\DeclareRobustCommand{\triang2}[2]{
	\left(
		\begin{matrix}
		#1_{1,1}#2	& 0 		& \cdots	 & 0 \\
		\vdots	 	& \ddots	& \ddots 	 & \vdots \\
		\vdots 		& 	 	& \ddots 	 & 0 \\
		#1_{n,1}#2	& \cdots        & \cdots	 & #1_{n,n}#2\\
		\end{matrix}
	\right) 
}
\DeclareRobustCommand{\figref}[1]{\mbox{ Fig. \ref{#1}}}
\DeclareRobustCommand{\defref}[1]{\mbox{ Definition \ref{#1}}}
\DeclareRobustCommand{\enumref}[1]{\mbox{\ref{#1}}}
\DeclareRobustCommand{\appendixref}[1]{\mbox{ Appendix \ref{#1}}}
\DeclareRobustCommand{\tupelref}[2]{\mbox{\enumref{#1} \enumref{#2}}}
\DeclareRobustCommand{\eqref}[1]{\mbox{ (\ref{#1})}}
\DeclareRobustCommand{\carlos}[1]{{\color{green} Carlos #1}}
\DeclareRobustCommand{\citeme851}{({\color{red} control theory lecture, the guy has not replied yet for a better citation)}}

%%%%%%%%%%%%%%%%%%%%%%%%%%%%%%%%%%%%%%%%%%%%%%%%%%%%%%%%%%%%%%%%%%%%%%%%%%%%%%
%%% Begin of Document
%%%%%%%%%%%%%%%%%%%%%%%%%%%%%%%%%%%%%%%%%%%%%%%%%%%%%%%%%%%%%%%%%%%%%%%%%%%%%%

\begin{document}

%%%%%%%%%%%%%%%%%%%%%%%%%%%%%%%%%%%%%%%%%%%%%%%%%%%%%%%%%%%%%%%%%%%%%%%%%%%%%%
%%% Here starts the poster
%%%---------------------------------------------------------------------------
%%% Format it to your taste with the options
%%%%%%%%%%%%%%%%%%%%%%%%%%%%%%%%%%%%%%%%%%%%%%%%%%%%%%%%%%%%%%%%%%%%%%%%%%%%%%
% Define some colors

\definecolor{lightblue}{rgb}{.187,.613,.594}
%\definecolor{lightblue}{rgb}{0.145,0.6666,1}


\hyphenation{resolution occlusions}
%%
\newcommand{\numberofcolumns}{5}
\begin{poster}%
  % Poster Options
  {
  % Show grid to help with alignment
  grid=true,
  % Column spacing
  colspacing=1em,
  columns=5,
  % Color style
  bgColorOne=white,
  bgColorTwo=white,
  borderColor=lightblue,
  headerColorOne=black,
  headerColorTwo=lightblue,
  headerFontColor=white,
  boxColorOne=white,
  boxColorTwo=lightblue,
  % Format of textbox
  textborder=roundedleft,
  % Format of text header
  eyecatcher=true,
  headerborder=closed,
  headerheight=0.15\textheight,
%  textfont=\sc, An example of changing the text font
  headershape=roundedright,
  headershade=shadelr,
  headerfont=\Large, %\textsc, %Sans Serif
  textfont={\setlength{\parindent}{1.5em}},
  boxshade=plain,
%  background=shade-tb,
  background=plain,
  linewidth=2pt
  }
  % Eye Catcher
  {\includegraphics[height=4.5em]{../../../../../../Administration/Logos/EmmyNoether}
  } 
  % Title
  {Application of Input to State Stability to ecological reservoir models}
  % Authors
  {
 {\bf  Markus M\"uller, Carlos A. Sierra}   \\
  {\small  <csierra@bgc-jena.mpg.de> } %\\ 
%  Independent Research Group \\
%  Theoretical Ecosystem Ecology
  }
  % University logo
  {% The makebox allows the title to flow into the logo, this is a hack because of the L shaped logo.
%   \includegraphics[height=5em]{Minerva} \\
   \includegraphics[height=2em]{BGClogo}
  }

  %%%%%%%%%%%%%%%%%%%%%%%%%%%%%%%%%%%%%%%%%%%%%%%%%%%%%%%%%%%%%%%%%%%%%%%%%%%%%%
  %%% Now define the boxes that make up the poster
  %%%---------------------------------------------------------------------------
  %%% Each box has a name and can be placed absolutely or relatively.
  %%% The only inconvenience is that you can only specify a relative position 
  %%% towards an already declared box. So if you have a box attached to the 
  %%% bottom, one to the top and a third one which should be in between, you 
  %%% have to specify the top and bottom boxes before you specify the middle 
  %%% box.
  % since in this example we have several boxes with the same horizontal alignment and widht
  % we set some new variables
  \newcommand{\leftspan}{2}
  \newcommand{\leftcol}{0}

  \newcommand{\rightspan}{\numexpr (\numberofcolumns - \leftspan) \relax}
  \newcommand{\rightcol}{\leftspan} % since colums start with 0 colum 2 is the third....
 

  % To be able to position boxex relative to  ('above') the footer  we 
  % have to define it first (as mentioned above more generally)
  %%%%%%%%%%%%%%%%%%%%%%%%%%%%%%%%%%%%%%%%%%%%%%%%%%%%%%%%%%%%%%%%%%%%%%%%%%%%%%
  % footerbox 
    \headerbox{
    }{
      name=footerbox,
      span=\numberofcolumns,
      column=0,
      textborder=none,
      headerborder=none,
      boxheaderheight=0pt,
      %below=combi,
      %above=bibliography
      above=bottom
    }{
    	
% vim:set ff=unix expandtab ts=2 sw=2:
%%%%%%%%%%%%%%%%%%%%%%%%%%%%%%%%%%%%%%%%%%%%%%%%%%%%%%%%%%%%%%%%%%%%%%%%%%%%%%
\noindent
\begin{itemize}
  \item
  Autonomous concepts like steady state are clearly insufficient for the analysis of non-autonomous systems.
  \item 
  Nonautonomous techniques are often restricted to linear systems.
  \item
  We propose Input to State Stability (ISS) as
  candidate for the necessary generalization of the established analysis with
  respect to equilibria or invariant sets for autonomous systems, 
  \item 
  In the just puplished  paper \cite{MuellerSierra2017TE} 
  we showed: 
  \begin{itemize}
  \item 
    How ISS generalizes existent concepts formerly only available for Linear Time Invariant (LTI) 
  and Linear Time Variant (LTV) systems to the nonlinear case. 
  \item 
    Exmaples applying it to reservoir models typical for element cycling in
  ecosystem, e.g. in soil organic matter decomposition.  
  \end{itemize}
\end{itemize}

    }
  %%%%%%%%%%%%%%%%%%%%%%%%%%%%%%%%%%%%%%%%%%%%%%%%%%%%%%%%%%%%%%%%%%%%%%%%%%%%%%

  % first column  
    % top
	  \headerbox{
      Challenge
    }{
      name=Challenge,
      span=\leftspan,
      column=\leftcol, 
      %row=0
    }{
	  	
% vim:set ff=unix expandtab ts=2 sw=2:
\noindent
Many models in ecology and biogeochemistry, in particular models of the global
carbon cycle, can be generalized as systems of non-autonomous ordinary
differential equations (ODEs). For many applications, it is important to
determine the stability properties for this type of systems, but most methods
available for autonomous systems are not necessarily applicable for the
non-autonomous case.  We discuss here stability notions for non-autonomous
nonlinear models represented by systems of ODEs explicitly dependent on time
and a time-varying input signal.  
Is there a stability concept that is:
\begin{enumerate}
	\item
	broad enough to encompass these models
	\item
	rigorous enough to be proved analytically 
	\item
	interpretable in ecologically meaningful terms ?
\end{enumerate}

	  }
	  %%%%%%%%%%%%%%%%%%%%%%%%%%%%%%%%%%%%%%%%%%%%%%%%%%%%%%%%%%%%%%%%%%%%%%%%%%%%%%
    % bottom 
	  \headerbox{
      Results I, ISS as generalization of available stability concepts
    }{
      name=generalsoilmodel,
      span=\leftspan,
      column=\leftcol, 
      %row=2, 
      %above=bottom
      above=footerbox
    }{
	  	
% vim:set ff=unix expandtab ts=2 sw=2:
%\begin{multicols}{2}
\begin{center}
\includegraphics[width=\columnwidth,clip=true,trim=4.5cm 1cm 3.2cm 2cm]{Fig1.pdf}
\end{center}

%\columnbreak
\noindent
The graph shows different stability concepts one could try to establish for the
general soil model mentioned above depending on properties of its components 
$\mathbf{I},\mathbf{T}$ and $ \mathbf{N}$. The hardest to prove is Input to
State Stability for time varying systems (ISStv) in the lower left corner.  It
turns out that ISStv also generalizes all the other concepts mentioned:
\begin{itemize}
\item 
In the case of Linear Time Invariant (LTI) systems  ISS follows from the properties of the matrix.(eigenvalues)
\item 
In the case of Linear Time Variant (LTV) systems it can be established if sufficient information about the 
state transition operator allows to prove uniform asymptotic stability UAS.
\item
For input free system (on the blue right-hand side) ISS reduces to Global Asymptotic Stability (GAS)
\item
\dots

\end{itemize}
%\end{multicols}

	  }
	  %%%%%%%%%%%%%%%%%%%%%%%%%%%%%%%%%%%%%%%%%%%%%%%%%%%%%%%%%%%%%%%%%%%%%%%%%%%%%%
    % middle
	  \headerbox{
      Example: general soil model
    }{
      name=general,
      span=\leftspan,
      column=\leftcol,
      %row=2, 
      below=Challenge,
      above=generalsoilmodel
    }{
	  	
% vim:set ff=unix expandtab ts=2 sw=2:
\begin{multicols}{2}
    		\[
		\mathbf{\dot{C}}= \bm{I}(t) + {\bf T}(\mathbf{C},t) \cdot {\bf N}(t, \bm{C}) \cdot \bm{C}(t)
    		\]
    		\begin{equation*}	
    		\label{structCond}
    		\begin{array}{lcl}	
    		N_{i,i}(\mathbf{C},t) 		&\ge& 	 0 \quad \forall i \\
    		T_{i,i}(\mathbf{C},t) 		&=& 	 -1 \quad \forall i \\
    		T_{i,j}(\mathbf{C},t) 		&\ge& 	 0 \quad \forall i \ne j \\
    		\sum_i T_{i,j}(\mathbf{C},t) 	&=  &	 1\quad \forall j 
    		\end{array}	
    		\end{equation*}	
    		This model structure generalizes most SOM decomposition models with any arbitrary number of pools, including those describing nonlinear interactions among state variables. It enforces mass balance and substrate dependence of decomposition, and it is flexible enough to describe:
		\begin{enumerate}
		\item Heterogeneity of decomposition rates
		\item Transformations of organic matter
		\item Environmental variability effects
		\item Organic matter interactions
		\end{enumerate}
    \columnbreak
		Examples for nonlinear models are:
		\begin{enumerate}
			\item Exoenzyme models \citep{Schimel,Sinsabaugh}
			\item AWB \citep{Allison}
			\item Bacwave \citep{Zelenev}
			\item MEND \citep{WangMEND}
			\item Manzoni \citep{Manzoni07}
		\end{enumerate}
		Also linear models fit into the general framework 
		\begin{enumerate}
			\item Henin's model \citep{HeninDupuis, Henin}
			\item ICBM \citep{AndrenKatterer}
			\item RothC \citep{Jenkinson, Coleman} 
			\item Century \citep{Parton} 
			\item Fontaine 1-4 \citep{Fontaine}
		\end{enumerate}
\end{multicols}
One general concept to encompass especially  these nonlinear  models is clearly desirable.

	  }
  %%%%%%%%%%%%%%%%%%%%%%%%%%%%%%%%%%%%%%%%%%%%%%%%%%%%%%%%%%%%%%%%%%%%%%%%%%%%%%
  % second column  
    % top
      \headerbox{
        Results II, ISS like behavior and proof for example system
      }{
        name=combi, 
        column=\rightcol, 
        span=\rightspan,
        row=0
      }{
    	  
% vim:set ff=unix expandtab ts=2 sw=2:
%\begin{columns}
%  \newlength{\lc}
%  \setlength{\lc}{0.4\textwidth}
%  \newlength{\rc}
%  \setlength{\rc}{\dimexpr(\textwidth-\lc) \relax}
%  \begin{column}{\lc}
\begin{multicols}{2}
  The graphs show the reactions of a prototypical class of nonlinear two pool soil models to a disturbing time varying signal. 
  This model is a technically simple place holder for ecologically motivated nonlinear systems like the soil models mentioned above to be analyzed in the future. It is given by:\\
  \begin{eqnarray*}
  \dot{C}_x=I_{x}(t)  - \left(C_{x}^{2} + C_{x}\right) \operatorname{k_{x}}{\left (t \right )}\\
  \dot{C}_y=I_{y}(t)  - \left(C_{y}^{2} + C_{y}\right) \operatorname{k_{y}}{\left (t \right )}
  \end{eqnarray*}
  where $C_x,C_y$ are the carbon contents of two unconnected pools and the bounded  periodic functions $k_x(t) $ and $k_y(t) $ with:\\ 
  \begin{eqnarray*}
  k_{x_{min}}\le  k_x(t) \le k_{x_{max}} \\k_{y_{min}} \le  k_y(t) \le k_{y_{max}}
  \end{eqnarray*}
  describe the seasonal changes in decomposition speed.
  e.g.:\\ 
  \begin{eqnarray*}
  k_x=\frac{k_{xmax}}{2} + \frac{k_{xmin}}{2} + \frac{1}{2} \left(k_{xmax} - k_{xmin}\right) \sin{\left (4 t \right )}\\k_y=\frac{k_{ymax}}{2} + \frac{k_{ymin}}{2} + \frac{1}{2} \left(k_{ymax} - k_{ymin}\right) \sin{\left (4 t \right )}
  \end{eqnarray*}
  The system can have a fixed point: 
  $$ \mathbf{C}_f= \begin{pmatrix} C_{fx} \\ C_{fy} \end{pmatrix} $$
  if the input streams have the same period and phase as the decomposition rates. For constant input streams it stays in a predictable region (an invariant set in the phase plane) \\ 
  \begin{eqnarray*}
  \mathbf{I}_0(t)=\left(
      \begin{pmatrix}
        \left( C_{fx}^{2} + C_{fx} \right) \operatorname{k_{x}}{\left (t \right )}\\
        \left( C_{fy}^{2} + C_{fy} \right)  \operatorname{k_{y}}{\left (t \right )}
      \end{pmatrix}
    \right)
  \end{eqnarray*}
  The fixpoint would be.
  \[
  \mathcal{A}=\{\mathbf{C}_f\}
  \]
  We will now disturb both mass influxes individually by perturbations $u_x(t),u_y(t)$ and get: 
  \begin{eqnarray*}
  \dot{C}_x=I_{0 x}(t) + u_{x}(t) - \left(C_{x}^{2} + C_{x}\right) \operatorname{k_{x}}{\left (t \right )}\\
  \dot{C}_y=I_{0 y}(t) + u_{y}(t) - \left(C_{y}^{2} + C_{y}\right) \operatorname{k_{y}}{\left (t \right )}
  \end{eqnarray*}
  The plots show the typical behavior of an ISS system: The changes in the state variables will asymptotically converge to a region of stability around an invariant set, whose size is a monotone function of the size of the disturbance (denoted by $|u|_{\infty}$).
  For this particular system we proved the ISS property rigorously. The proof relies on the construction of an ISS Lyapunov function whose choice is \emph{not determined but  inspired} by a property of the system interpretable in ecologically terms. Expressed casually: "The  system can counteract supply changes fast enough".
  This situation seems to be typical: The problem of establishing ISS for e.g. all the $\mathbf{I},\mathbf{T},\mathbf{N}$ models based on the ecologic principles they follow , is too hard.
  But bio-chemical of biophysical restrictions could provide clues to ISS proofs for a particular system.
  
%  \end{column}
%  \begin{column}{\rc}
    \includegraphics[width=\columnwidth,clip=true,trim=1.0cm 0cm 2cm 1cm]{combiPlot2.pdf}
    \begin{enumerate}
    	\item
    	The four plots on the top show the disturbances.
    	\item
    	The next four plots in the middle show the effect of this disturbances on the solutions for a system with fixed point. 
    	\item
    	The next four plots in the middle show the effect of this disturbances on the solutions for the system which no longer has a fixed point, but at least an invariant set, the dark blue square in the middle.
    \end{enumerate}
 % \end{column}
%\end{columns}
\end{multicols}
\vspace{1cm}

      }
    %%%%%%%%%%%%%%%%%%%%%%%%%%%%%%%%%%%%%%%%%%%%%%%%%%%%%%%%%%%%%%%%%%%%%%%%%%%%%%
	 % \headerbox{
   %   Bibliography 
   % }{
   %   name=bibliography,
   %   span=\rightspan,
   %   column=\rightcol, 
   %   %row=2, 
   %   above=bottom
   % }{
   %   {
   %     \tiny
   %     \begin{multicols}{3}
   %       %\nobibliography{GeneralModel} % the citations are there but no Referenc section
   %       %\renewcommand{\section}[2]{}% to get rid of the References Header
   %       %\documentclass[final,hyperref={pdfpagelabels=false}, professionalmath, mathserif, 11pt]{beamer}
\usepackage{grffile}
\mode<presentation>{\usetheme{BGC_retreat}}
\usepackage[english]{babel}
\usepackage[utf8]{inputenc}
\usepackage{graphicx}
\usepackage[round]{natbib}

 
\def\mytitle{SoilR and a General SOM Decompositon Model }
%\def\mysubtitle{Application to Stability}
\def\myauthor{Markus Müller \quad Carlos Sierra }
\def\myfooterleft{
    \begin{minipage}[T]{.95\textwidth}

\begin{thebibliography}{15}
%\providecommand{\natexlab}[1]{#1}
%\providecommand{\url}[1]{\texttt{#1}}
%\expandafter\ifx\csname urlstyle\endcsname\relax
%  \providecommand{\doi}[1]{doi: #1}\else
%  \providecommand{\doi}{doi: \begingroup \urlstyle{rm}\Url}\fi


\bibitem[Sierra et~al.(2012)Sierra, M\"uller, and Trumbore]{SierraGMD}
C.~A. Sierra, M.~M\"uller, and S.~E. Trumbore.
\newblock Models of soil organic matter decomposition: the {SoilR} package,
  version 1.0.
\newblock \emph{Geosci. Model Dev.}, 5\penalty0 (4):\penalty0 1045--1060, 2012.
\newblock GMD.

\bibitem[Sierra et~al.(2014)Sierra, M\"uller, and Trumbore]{SierraGMD14}
C.~A. Sierra, M.~M\"uller, and S.~E. Trumbore.
\newblock Modeling radiocarbon dynamics in soils: {SoilR} , version 1.1.
\newblock \emph{Geosci. Model Dev.}, 7\penalty0 (7):\penalty0 1919--1931, 2014.
\newblock GMD.

\end{thebibliography}
     \end{minipage}
}
\def\myfooterright{
    \begin{minipage}[T]{.95\textwidth}
	\begin{minipage}[c]{5cm}
	\includegraphics[height=4cm]{MarkusMueller.jpg}
	\end{minipage}
	\begin{minipage}[c]{10cm}
	Markus Müller
	\end{minipage}
	\begin{minipage}[c]{5cm}
	\includegraphics[height=4cm]{Sierra.jpg}
	\end{minipage}
	\begin{minipage}[c]{10cm}
	Carlos Sierra
	\end{minipage}
	\begin{minipage}[c]{5cm}
	\includegraphics[height=4cm]{qrcode.png}
	\end{minipage}
     \end{minipage}
\vspace{1cm}
}

\usepackage{setspace}
\setstretch{1.25}

\usepackage{bm}
\usepackage{tikz}
\usetikzlibrary{arrows,backgrounds,calc,chains,fadings,fit,graphs,mindmap,patterns,positioning,quotes,scopes,shapes.geometric,trees} 

\definecolor{yellow}{rgb}{1,1,0}
\definecolor{green}{rgb}{0,1,0}
\definecolor{GeneralModel}{rgb}{0.709803921568627,0.870588235294118,0.63921568627451}
\definecolor{TheoreticalAnalysis}{rgb}{1,1,0.8}
\definecolor{SoilPy}{rgb}{0.63921568627451,0.83921568627451,0.8}
\definecolor{SoilR}{rgb}{0.8,0.8,1}

\colorlet{lightblue}{blue!25}
\colorlet{darkblue}{blue!75}
\colorlet{lightpink}{red!12}
\colorlet{darkred}{red!75}
\colorlet{red}{red!50}
\colorlet{LTICTBg}{lightpink}
\colorlet{GASDSTBg}{blue}
\colorlet{LDSTBg}{yellow!30!blue!30}

%%%%<
%\usepackage{verbatim}
%\usepackage[active,tightpage]{preview}
%\PreviewEnvironment{tikzpicture}
%\setlength\PreviewBorder{5pt}%
%%%%>

%node[pos=0.5]{yes}

\newcommand{\captionfont}{\footnotesize}
\newcommand{\SoilR}{\textsc{SoilR}}
\newcommand{\RC}{$\Delta^{14}$C}
\newcommand{\figsize}{0.108}
\DeclareRobustCommand{\diag3}[3]{
	\left(
		\begin{matrix}
		#1	& 0  		& \cdots	 & 0 \\
		0 	& \ddots	& \ddots 	 & \vdots \\
		\vdots 	& \ddots 	& \ddots 	 & 0 \\
		0 	& \cdots        & 0		 & #3\\
		\end{matrix}
	\right) 
}
\DeclareRobustCommand{\triang2}[2]{
	\left(
		\begin{matrix}
		#1_{1,1}#2	& 0 		& \cdots	 & 0 \\
		\vdots	 	& \ddots	& \ddots 	 & \vdots \\
		\vdots 		& 	 	& \ddots 	 & 0 \\
		#1_{n,1}#2	& \cdots        & \cdots	 & #1_{n,n}#2\\
		\end{matrix}
	\right) 
}
\DeclareRobustCommand{\figref}[1]{\mbox{ Fig. \ref{#1}}}
\DeclareRobustCommand{\defref}[1]{\mbox{ Definition \ref{#1}}}
\DeclareRobustCommand{\enumref}[1]{\mbox{\ref{#1}}}
\DeclareRobustCommand{\appendixref}[1]{\mbox{ Appendix \ref{#1}}}
\DeclareRobustCommand{\tupelref}[2]{\mbox{\enumref{#1} \enumref{#2}}}
\DeclareRobustCommand{\eqref}[1]{\mbox{ (\ref{#1})}}
\DeclareRobustCommand{\carlos}[1]{{\color{green} Carlos #1}}
\DeclareRobustCommand{\citeme851}{({\color{red} control theory lecture, the guy has not replied yet for a better citation)}}

\pgfkeys{/TNIoft/.code={$\mathbf{I}(t)$, $\mathbf{N}(t)$, $\mathbf{T}(t)$?}}
\pgfkeys{/fofC/.code={$\TM(\CM)$,$\NM(\CM)$ ?}} 
\pgfkeys{/foft/.code={$\TM(t),\NM(t)$?}} 
\pgfkeys{/ipsi/.code={$\IM(t)=\xi(t)\IM$?}} %the extra { } are needed to escape =
\pgfkeys{/iper/.code={$\IM(t)$ periodic?}}
\pgfkeys{/iconst/.code={$\IM(t)$ const.?}}
\pgfkeys{/stab/.code={S}}

\tikzset{
  every node/.style={font=\sfs},%,text depth=.2ex},
  test/.style={diamond,draw,minimum width=.5em,text width=2em,aspect=1,align=center  },
  comment/.style={rectangle,text width=3cm,minimum width=1cm,align=center},
  decision/.style = {test}, 
  sact/.style    = { rectangle, draw=blue, thick, 
                        fill=blue!20, text width=9em, 
                        rounded corners, minimum height=1cm},
  description/.style    = { rectangle, draw=black, thick, 
                        fill=black!20, text width=12em, text centered,
                        rounded corners, minimum height=.5em},
  ce/.style    = { circle, draw=black, thick, 
                        fill=blue!20, text width=1em, text centered,
                        rounded corners, minimum height=1em},
  cst/.style={draw, black, circle,text width=1em,child anchor=south},
  cex/.style={draw=black,fill=white, circle,text width=1em,minimum height=1em,align=center,text=red,node font=\bf},
  ceok/.style={draw=black,fill=white, circle,text width=1em,minimum height=1em,align=center,text=green,node font=\bf},
  line/.style     = { draw, thick, ->, shorten >=2pt }
}

%\setbeamersize{text margin left=5pt,text margin right=5pt}
%\newtheorem{theorem}{Theorem} 
\newtheorem{Def}{Definition:}
%\newtheorem{lemma}{Lemma:}

%%%%%%%%%%%%%%%%%%%%%%%%%%%%%%%%%%%%%%%%%%%%%%%%%%%%%%%%%%%%%%%%%%%%%%%%%%%%%%%%%%%%%%
\begin{document}
\graphicspath{ {./images/} }
\begin{frame}
%	\begin{center}
%\input{uas.tex}
%	\end{center}
\begin{columns}
  % ------------
  % FIRST COLUMN
  % ------------
  
  \begin{column}{.48\textwidth}
    \begin{minipage}[T]{.95\textwidth}
      %%%%%%%%%%%%%%%%%%%%%%%%%%%%%%%%%%%%%%%%%%%%%%%%%%%%%%%%%%%%%%%%%%%%%%%%%%%%%%%%%%%%%%
      \setbeamercolor*{block title}{bg=darkgray}
      \begin{block}{Overview}
	\begin{figure}
	  \includegraphics[height=20cm]{soilrMindMap.pdf}
	  \caption{The General Model concept as central synthesis tool and product of theory and software development}
	\end{figure}
      \end{block}
    %%%%%%%%%%%%%%%%%%%%%%%%%%%%%%%%%%%%%%%%%%%%%%%%%%%%%%%%%%%%%%%%%%%%%%%%%%%%%%%%%%%%%%
    \setbeamercolor*{block title}{bg=GeneralModel}
    \begin{block}{The General SOM Decomposition Model}
    \begin{columns}[b]
    \column{.5\textwidth}
    		\[
		\mathbf{\dot{C}}= \bm{I}(t) + {\bf T}(\mathbf{C},t) \cdot {\bf N}(t, \bm{C}) \cdot \bm{C}(t)
    		\]
    		\begin{equation*}	
    		\label{structCond}
    		\begin{array}{lcl}	
    		N_{i,i}(\mathbf{C},t) 		&\ge& 	 0 \quad \forall i \\
    		T_{i,i}(\mathbf{C},t) 		&=& 	 -1 \quad \forall i \\
    		T_{i,j}(\mathbf{C},t) 		&\ge& 	 0 \quad \forall i \ne j \\
    		\sum_i T_{i,j}(\mathbf{C},t) 	&=  &	 1\quad \forall j 
    		\end{array}	
    		\end{equation*}	
    		This model structure generalizes most SOM decomposition models with any arbitrary number of pools, including those describing nonlinear interactions among state variables. It enforces mass balance and substrate dependence of decomposition, and it is fexible enough to describe:
		\begin{enumerate}
		\item Heterogeneity of decomposition rates
		\item Transformations of organic matter
		\item Environmental variability effects
		\item Organic matter interactions
		\end{enumerate}

    \column{.5\textwidth}
		Examples for nonlinear models are:
		\begin{enumerate}
			\item Exoenzyme models \citep{Schimel,Sinsabaugh}
			\item AWB \citep{Allison}
			\item Bacwave \citep{Zelenev}
			\item MEND \citep{WangMEND}
			\item Manzoni \citep{Manzoni07}
		\end{enumerate}
		Also linear models fit into the general framework 
		\begin{enumerate}
			\item Henin's model \citep{HeninDupuis, Henin}
			\item ICBM \citep{AndrenKatterer}
			\item RothC \citep{Jenkinson, Coleman} 
			\item Century \citep{Parton} 
			\item Fontaine 1-4 \citep{Fontaine}
		\end{enumerate}
    \end{columns}
    \end{block}
      %%%%%%%%%%%%%%%%%%%%%%%%%%%%%%%%%%%%%%%%%%%%%%%%%%%%%%%%%%%%%%%%%%%%%%%%%%%%%%%%%%%%%%
      \setbeamercolor*{block title}{bg=SoilPy}
      \begin{block}{
	  \includegraphics[height=1.5cm]{logo_soilpy} 
	  \hspace{1cm}
	  python Software, not yet published 
	  }
	\begin{enumerate}
		\item A dynamic  catalog to \emph{reproduce} the symbolic math representations of all the above 
		mentioned models in terms of the general model structure.
		\item Tools to check the validity of the models (properties of $\mathbf{T}$ and $\mathbf{N}$)
		\item Tools to verify symbolically available fixed points 
		\item Tools to analyze the stability of those fixed points
		\item Tools to compute and verify fixed points numerically
		\item Tools to visualize 2-pool and 3-pool models in the phase space 
		\item Symbolic coordinate transformations to facilitate stability analysis
	\end{enumerate}
The following example \citep{Manzoni07} is copied from the (shortened) \LaTeX \hspace{1ex}  report of the software:\\ 
$\frac{d}{d{t}}\left(\begin{matrix}C_{S}\\C_{B}\end{matrix}\right)=\left(\begin{matrix}ADD - \frac{C_{B} C_{S} k_{S}}{C_{S} + K_{m}} + C_{B} k_{B}\\\frac{C_{B} C_{S} k_{S}}{C_{S} + K_{m}} \left(- r + 1\right) - C_{B} k_{B}\end{matrix}\right) $
translates to 
$ \mathbf{\dot{C}}=\mathbf{I + T\cdot N(C)\, C} $\\
with: $\mathbf{\dot{C}}=\frac{d}{d{t}}\left(\begin{matrix}C_{S}\\C_{B}\end{matrix}\right)$,
$\mathbf{T}=\left(\begin{matrix}-1&  1\\- r + 1 & -1\end{matrix}\right)$,
$\mathbf{N}=\left(\begin{matrix}\frac{C_{B} k_{S}}{C_{S} + K_{m}} & 0\\0 & k_{B}\end{matrix}\right) \text{}$,
$\mathbf{I}=\left(\begin{matrix}ADD\\0\end{matrix}\right)$.\\ 
Checking alphas:
$ r_{0}=-\sum_{k=1}^{n}T_{k,0}=r $ ,$ r_{1}=-\sum_{k=1}^{n}T_{k,1}=0 $.\\
Checking suggested symbolic fixed point(s):
$ \mathbf{\tilde{C}}=\left(\begin{matrix}\frac{K_{m} k_{B}}{- k_{B} + k_{S} \left(- r + 1\right)}\\\frac{ADD}{k_{B} r} \left(- r + 1\right)\end{matrix}\right)$, substitute:
$ \mathbf{\dot{C}}(\mathbf{\tilde{C}})=\left(\begin{matrix}0\\0\end{matrix}\right) $\\
With numeric parameters:
$ \mathbf{\dot{C}}=\left(\begin{matrix}- \frac{1.8 \cdot 10^{-5} C_{B} C_{S}}{C_{S} + 900} + 0.007 C_{B} + 3.3\\\frac{7.2 \cdot 10^{-6} C_{B} C_{S}}{C_{S} + 900} - 0.007 C_{B}\end{matrix}\right) $
	\begin{figure}
	  \includegraphics[height=11cm]{posfp.pdf}
	  \includegraphics[height=11cm]{3D-OscillatingStayPositiveExample}
	  \caption{On the left is the phase plane plot of a model with a fixed point. On the left is a phase space plot of a linear 3 pool feedback model showing the plane of damped  oscillations and the boundaries of the positive octant.
	  }
	\end{figure}
      \end{block}
    \end{minipage}
  \end{column}
  
    %%%%%%%%%%%%%%%%%%%%%%%%%%%%%%%%%%%%%%%%%%%%%%%%%%%%%%%%%%%%%%%%%%%%%%%%%%%%%%%%%%%%%%
  % -------------
  % SECOND COLUMN
  % -------------
  
  
  \begin{column}{.48\textwidth}
    \begin{minipage}[T]{.95\textwidth}
      %%%%%%%%%%%%%%%%%%%%%%%%%%%%%%%%%%%%%%%%%%%%%%%%%%%%%%%%%%%%%%%%%%%%%%%%%%%%%%%%%%%%%%
      \setbeamercolor*{block title}{bg=TheoreticalAnalysis}
      \begin{block}{Theoretical Stability Analysis}
      The unified description of many models in one formula offers: 
      \begin{enumerate}
	\item 
	A strict and unambiguous definition and  treatment of different concepts, e.g. \emph{Stability}. 
	\item
	Access to existing and relevant mathematical literature by the right keywords. 
	\item 
	Means to classify models w.r.t. mathematical properties and thereby:
	\item 
	A way to decide which questions to ask about which models, thereby providing both of the following: 
	\item 
	Inspiration for generalization, a focal point to answer many questions about many models at once, or
	\item 
	Protection against futile quests by discovering unanswerable questions, e.g. \emph{steady} states for time\emph{variant} systems... 
      \end{enumerate}
	\begin{figure}
	\newlength{\ab}
\setlength{\ab}{3em}
\newlength{\lsdone}
\newlength{\ldone}
\newlength{\lsdtwo}
\newlength{\lsdthree}
\newlength{\lsdfour}
\setlength{\lsdone}{.9\textwidth}
\setlength{\ldone}{.13\lsdone}
\setlength{\lsdtwo}{    .5\lsdone}
\setlength{\lsdthree}{ .25\lsdone}
\setlength{\lsdfour}{ .125\lsdone}
\newlength{\radius}
\setlength{\radius}{.5\textwidth}
\newlength{\subwidthlr}
\setlength{\subwidthlr}{.45\textwidth}

	    \begin{tikzpicture}
  [ 
    %mindmap,
    every node/.style={fill=white,execute at begin node=\hskip0pt,font=\tiny},
    level 1/.append style={level distance=\ldone,sibling distance=\lsdone},
    level 2/.append style={sibling distance=\lsdtwo},
    level 3/.append style={sibling distance=\lsdthree},
    level 4/.append style={sibling distance=\lsdfour},align=center
  ]
  \node[cst](cst) {root}
  %\pgfpathcircle{\pgfpointanchor{x}{north}}{2pt}
    child{
      node[test](TNIoft) {\pgfkeys{/TNIoft}} 
      %child[concept color=red]{
      %    node[proc] {BiBo, CiCo or ISS } 
          child{
            node[test](CLorNL){\pgfkeys{/fofC}}
            edge from parent[parent anchor=west,child anchor=north, edge from parent path={\mypath}]
                child{
                  node[test](NLTIorNLTV){\pgfkeys{/foft}}
                  child{
                    node[test]{tv ISS proof}
                    child{
                      node[ce](mark){}
                      edge from parent[parent anchor=west,child anchor=north, edge from parent path={\mypath}]
	    	    } 
		    child[missing]{}
                    edge from parent[parent anchor=west,child anchor=north, edge from parent path={\mypath}]
	    	  } 
                  child{
                    node[test]{ISS proof}
                    child{
                      node[ce] {}
                      edge from parent[parent anchor=west,child anchor=north, edge from parent path={\mypath}]
	    	    } 
		    child[missing]{}
                    edge from parent[parent anchor=east,child anchor=north, edge from parent path={\mypath}]
	    	  } 
                    edge from parent[parent anchor=west, edge from parent path={\mypath}]
		}
              child{
                node[test](LTIorLTV) {\pgfkeys{/foft}} 
                child{
                  node[test](LTV){UAS proof}
                  child{
                    node[ce]{}
                    edge from parent[parent anchor=west,child anchor=north, edge from parent path={\mypath}]
	    	  } 
		  child[missing]{}
                  edge from parent[parent anchor=west,child anchor=north, edge from parent path={\mypath}]
	    	} 
                child{
                  node[ce](LTI){}
                  edge from parent[parent anchor=east,child anchor=north, edge from parent path={\mypath}]
	    	} 
                edge from parent[parent anchor=east,edge from parent path={\mypath}]
              }
	  node[left,fill=none]{yes}
	  }
	  %%%%%%%%%%%%%%%%%%%%%%%%%%%%%%%%%%%%%%%%%%%%%%%%%%%%%%%%%
          child{
            node[test](DLorNL){\pgfkeys{/fofC}}
              child{
                  node[test](GAS){GAS proof}
                  child{
                    node[ce]{}
                    edge from parent[parent anchor=west,child anchor=north, edge from parent path={\mypath}]
	    	  } 
                  child{
                    node[test]{AS proof per FP}
                    child{
                      node[ce](ASend) {}
                      edge from parent[parent anchor=west,child anchor=north, edge from parent path={\mypath}]
	    	    } 
		    child[missing]{}
                    edge from parent[parent anchor=east,child anchor=north, edge from parent path={\mypath}]
	    	  } 
                    edge from parent[parent anchor=west, edge from parent path={\mypath}]
		}
              child{
                node[ce](LTC) {} 
                edge from parent[parent anchor=east,edge from parent path={\mypath}]
              }
            edge from parent[parent anchor=east,child anchor=north, edge from parent path={\mypath}]
	    node[above,fill=none]{no}
	  }
    };
  %\begin{scope}[every annotation/.style={fill=black!40}]
  %	\node [annotation, above] at (CLorNL.north east) {
  %	{Control Theory}
  %};
  %\end{scope}
  \begin{pgfonlayer}{background}
    %\clip (-1.5,-5) rectangle ++(4,10);
    % The large rectangles:
    % first find the coordinates of the boundaries
    
    \fill [pink] (cst) rectangle (mark);
    \fill [lightblue] let \p1=(mark),\p2=(mark) in
    	(cst) rectangle (-\x1,\y2); 
    % level 2rectangles:
    \fill [red] (CLorNL) rectangle (mark);
    \fill[blue] let \p1=(cst),\p2=(mark) in  
    	(DLorNL) rectangle(\x1,\y2);
    % level 3rectangles:
    \fill[lightpink] let \p1=(cst),\p2=(mark) in  
    	(LTIorLTV) rectangle (\x1,\y2);
    \fill [darkred] (NLTIorNLTV) rectangle (mark);
    \fill [darkblue] let \p1=(DLorNL),\p2=(GAS) in
    	(ASend) rectangle (\x1,\y2); 
    
    \path(CLorNL) node[draw=pink,rectangle,fill=none,above=\ab]{\hfs Control Theory};
    \path(NLTIorNLTV) node[draw=none,rectangle,fill=none,above=\ab]{\hfs nonlinear};
    \path(LTIorLTV) node[draw=none,rectangle,fill=none,above=\ab]{\hfs linear};
    
    \path(GAS) node[draw=none,rectangle,fill=none,above=\ab]{\hfs nonlinear};
    \path(LTC) node[draw=none,rectangle,fill=none,above=\ab]{\hfs linear};
    
    \path(LTV) node[draw=none,rectangle,fill=none,above=\ab]{\hfs LTV};
    \path(LTI) node[draw=none,rectangle,fill=none,above=\ab]{\hfs LTI};
    \path(DLorNL) node[draw=lightblue,rectangle,fill=none,above=\ab, text width=.2\lsdone]{\hfs Dynamic System Theory};
  \end{pgfonlayer}
\end{tikzpicture} 

	  \caption{The graph shows one example about how the general model can guide decisions. Given an arbitrary instance of the general model, it  shows which  mathematically defined concept of \emph{stability} can possibly be established based on the mathematical properties of the model.}
	\end{figure}
      \end{block}
      %%%%%%%%%%%%%%%%%%%%%%%%%%%%%%%%%%%%%%%%%%%%%%%%%%%%%%%%%%%%%%%%%%%%%%%%%%%%%%%%%%%%%%
      \setbeamercolor*{block title}{bg=SoilR}
      \begin{block}{
	  \includegraphics[height=1.5cm]{logo_soilr} 
	  \hspace{1cm}
	  R Software,
	  public package on cran
	  }
	 SoilR is our publicly available code representation of the general model. It consists of both an extensive library of models and many computational tools to analyze them.  Its internal representation as object oriented framework helped the present form of the general model to emerge. In particular its purposes are the following:
	 \begin{enumerate}
	 \item Prove theoretical concepts by implementing them, exposing theoretical errors and ambiguities by the rigor required to run the code.
	 \item Disclose the underlying theoretical structure of the ecological system by the continuous restructuring required to integrate new non-contradictory features
	 \item Provide an easy to use framework to reproduce our scientific results
	 \item Provide a community tool that can be improved and modified by others, facilitating \emph{reproducible research}. 
	 \end{enumerate}
	 

	\begin{figure}
	  \includegraphics[height=9cm]{RothCfig} \hspace{1cm}
	  \includegraphics[height=9cm]{ModelFit.pdf}
	  \includegraphics[height=9cm]{NonlinearAtmosphericModelMeanAge}
	  %\includegraphics[height=10cm]{NonlinearAtmosphericModelMeanAge}
	  %\includegraphics[height=10cm]{MeanAgeSteady}
	  %\includegraphics[width=.3\textwidth]{CDI}
	  %\includegraphics[width=.5\textwidth]{PoolStructureEx.pdf}
	  \caption{
	  %\begin{minipage}{\textwidth}
	  Example applications of SoilR, 
	  left: carbon stocks, total and per pool, predicted by the Roth C model as functions of time; 
	   middle: example of a Bayesian parameter estimation combining SoilR with package FME.   
	   right: Mean age of carbon in a pool of an nonlinear model as function of time.
	  %\end{minipage}
	  }
	\end{figure}
	\begin{figure}
	  \includegraphics[height=7cm]{Reservoir_Peoples_Ages2}
	  \includegraphics[height=7cm]{MeanTransitTime}
	  %\includegraphics[width=.45\textwidth]{Reservoir_Peoples_Ages2}
	  %\includegraphics[width=.45\textwidth]{MeanTransitTime}
	  %\includegraphics[width=.45\textwidth]{NonlinearAtmosphericModelMeanAge}
	  %\includegraphics[width=.45\textwidth]{MeanAgeSteady}
	  \caption{Two intuitive definitions of mean age and transit time, implemented in SoilR for non-steady-state models. }
	\end{figure}
	We are currently developing a new method for calculating transit times and mean ages for models that are not in equilibrium and can also be nonlinear. 
	This is done with a Monte Carlo particle simulator that facilitates modeling of cycling times through a network of compartments. All particles in the system
	contain information about the time they enter the system, how long they have remained, and how old they are at the time they leave the system. 

      \end{block}

    \end{minipage}
  \end{column}
\end{columns}
	\vspace{3ex}
\end{frame}
\bibliography{GeneralModel}
\bibliographystyle{abbrvnat}
\end{document}

   %       \bibliographystyle{abbrvnat}
   %     \end{multicols}
   %   }
	 % }
   % %%%%%%%%%%%%%%%%%%%%%%%%%%%%%%%%%%%%%%%%%%%%%%%%%%%%%%%%%%%%%%%%%%%%%%%%%%%%%%
    % bottom of regular column
    % middle
    \headerbox{
      Conclusion
    }{
      name=Intro,
      span=\rightspan,
      column=\rightcol, 
      below=combi,
      %above=bottom
      above=footerbox
    }{
    	
% vim:set ff=unix expandtab ts=2 sw=2:
%%%%%%%%%%%%%%%%%%%%%%%%%%%%%%%%%%%%%%%%%%%%%%%%%%%%%%%%%%%%%%%%%%%%%%%%%%%%%%
\noindent
\begin{itemize}
  \item
  Autonomous concepts like steady state are clearly insufficient for the analysis of non-autonomous systems.
  \item 
  Nonautonomous techniques are often restricted to linear systems.
  \item
  We propose Input to State Stability (ISS) as
  candidate for the necessary generalization of the established analysis with
  respect to equilibria or invariant sets for autonomous systems, 
  \item 
  In the just puplished  paper \cite{MuellerSierra2017TE} 
  we showed: 
  \begin{itemize}
  \item 
    How ISS generalizes existent concepts formerly only available for Linear Time Invariant (LTI) 
  and Linear Time Variant (LTV) systems to the nonlinear case. 
  \item 
    Exmaples applying it to reservoir models typical for element cycling in
  ecosystem, e.g. in soil organic matter decomposition.  
  \end{itemize}
\end{itemize}

    }

%%%%%%%%%%
\end{poster}
\bibliography{GeneralModel} % for an extra page
\end{document}
