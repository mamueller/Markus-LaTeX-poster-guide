
% vim:set ff=unix expandtab ts=2 sw=2:
\begin{itemize}
    \item
      The built in environments of the beamer
      class (\verb1 columns1 and \verb1 block1)
      \alert{do not facilitate} a sensible automation 
      of vertical alignment of the boxes.
      If the columns are for instance aligned at the top the lowest boxes of different columns  will end at different vertical positions.
    \item   
      Basic functionality can be achieved by the presented approach of inflating the blocks with internal minipages.
    \item 
      The above code could be generelized in different ways: 
    \begin{itemize}
      \item 
        More columns and rows, different distribution of whitespace. 
      \item 
        Instead of enforcing a fixed vertical size of the header, the actual size could be measured and substracted from the
        parameter governing the size of the internal miniapage.
        The parameter could then be interpreted as an outer vertical size, which is more intuitive. 
    \end{itemize}
    \item
      Arbitrary positioning of boxes is not feasable.
    \item 
      Code achieving this level of generalty would better live in a package. 
    \item 
      Given the limitations and the complexity of the required workarounds we do not recommend the combination
      of \verb1 columns1 and \verb1 block1 environments as general approach to posters.
      The value of this example rather consists in the demonstration of the most basic approach to the alignment challenge,
      and its minimal dependencies on other packages.
    \item
      The approach should certainly not be mandatory for the use of a template.
      Such a template should allow the use of other approaches implemented by additional packages.
      We will present such packages in the following poster examples.
\end{itemize}
