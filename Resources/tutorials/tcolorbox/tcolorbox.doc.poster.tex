% !TeX root = tcolorbox.tex
% include file of tcolorbox.tex (manual of the LaTeX package tcolorbox)
\clearpage
\section{Library \mylib{poster}}\label{sec:poster}%
\tcbset{external/prefix=external/poster_}%

The main purpose of this library is to support creation of single page posters
with |tcolorbox|es.

A \refEnv{tcbposter} is a |tikzpicture| where |tcolorbox|es can be
placed in a column oriented manner using \refCom{posterbox} commands.
This base concept is more or less copied from the great |baposter| package.

The \mylib{raster} library, see \Fullref{sec:raster}, can produce
similar looking results and may be more appropriate
depending on the actual project.
\begin{itemize}
\item The \mylib{raster} library has a flow oriented concept, just like a
  convential text flow. The text flow (box flow) is a merely endless ribbon
  which gets broken into lines (and paragraphs) and the lines are broken
  into pages. \mylib{raster} shapes the boxes to convenient sizes to fill
  lines and pages in a pleasant way.
\item The \mylib{tcbposter} library supports a quite free placement of
  boxes inside a page.
  Basically, boxes are placed like |node|s are placed inside a |tikzpicture|.
  In contrast to \mylib{raster}, this is a \emph{single} page
  and not a flow of pages.
  The poster is divided into columns and rows.
  There is a more or less gentle force to use the columns (or spans of columns)
  for positioning and sizing while the row placement is completely optional.
\end{itemize}
The creation of this library was motivated by Ignasi.

\begin{marker}
Inside a |tikzpicture| there should be no embedded |tikzpicture|s.
This rule is violated by the \mylib{poster} library. Be aware that there
may be some unwanted interactions between the main |tikzpicture| and
the embedded ones inside the |tcolorbox|es.
\end{marker}

The library is loaded by a package option or inside the preamble by:
\begin{dispListing}
\tcbuselibrary{poster}
\end{dispListing}
This also loads the libraries
\mylib{skins}, see \Fullref{sec:skins},
\mylib{breakable}, see \Fullref{sec:breakable},
\mylib{magazine}, see \Fullref{sec:magazine}, and
\mylib{fitting}, see \Fullref{sec:fitting}.


%--------------------------
\subsection{Overview}\label{subsec:poster_overview}


\begin{tcolorbox}[base example,hyperurl={tcolorbox-tutorial-poster.pdf},title=Click me to see the tutorial]
You get the best overview of the \mylib{poster} library and its facilities,
if you look at the \textbf{Poster Tutorial} which is part of the |tcolorbox|
documentation:\par
\texttt{tcolorbox-tutorial-poster.pdf}
\end{tcolorbox}



\clearpage
%--------------------------
\subsection{Main Poster Environment}\label{subsec:poster_environment}

\begin{docEnvironment}[doc new=2017-07-03]{tcbposter}{\oarg{options}}
  This creates a |tikzpicture| environment with suitable additional
  settings defined by the given \meta{options}.
  Basically, \refCom{posterbox} and \refEnv{posterboxenv} are
  used to place |tcolorboxes| as nodes into the environment,
  but additional \tikzname\ code can also be used.
  As \meta{options} all |/tcb/posterset/| keys may be applied, namely:
\begin{itemize}
\item\refKey{/tcb/posterset/poster}: poster settings like columns, rows, sizes\ldots
\item\refKey{/tcb/posterset/coverage} and \refKey{/tcb/posterset/no coverage}:
  settings for a surrounding |tcolorbox| for background and margins.
\item\refKey{/tcb/posterset/boxes}: style of the |tcolorbox|es used for the poster.
\item\refKey{/tcb/posterset/fontsize}: scaling of used fonts.
\end{itemize}

\begin{exdispExample}{tcbposter}
  \begin{tcbposter}[
    poster = {showframe,height=10cm,spacing=2mm},
    boxes  = {beamer,colframe=blue!50!black,colback=blue!50,colupper=yellow!50},
  ]
  \posterbox{name=A,column=3,row=2}{My first box}
  \posterbox[adjusted title=Second box]
            {name=B,column=2,span=2,below=A}{My second box}
  \posterbox[adjusted title=Third box]
            {name=C,column=2,between=B and bottom}{My third box}
  \end{tcbposter}
\end{exdispExample}
\end{docEnvironment}

\clearpage
  Inside \refEnv{tcbposter}, there are several predefined \tikzname\ nodes.
  These nodes share a common \refKey{/tcb/poster/prefix} which is
  |TCBPOSTER@| by default. This prefix is used to discriminate the
  poster nodes from local nodes of any embedded |tikzpicture| environment.
  You will never need this prefix using \refCom{posterbox} and its
  placement options, but if you want to refer to a predefined node using
  pure \tikzname\ code.
  The predefined nodes (shown without prefix) are:
  \begin{itemize}
  \item|poster|: defines the bounding box of the poster (without the coverage).
  \item|top|: top position plus row spacing
  \item|bottom|: bottom position minus row spacing
  \item|middle|: vertical middle position
  \item|col1|, |col2|, \ldots: bounding box of column~1, column~2, \ldots
  \item|row1|, |row2|, \ldots: bounding box of row~1, row~2, \ldots
  \end{itemize}
  Further nodes are defined using the \refKey{/tcb/posterloc/name} option.

  \begin{marker}
  Never use a \refEnv{tcbposter} inside a \refEnv{tcbposter}.
  But, if you do anyway, use a different \refKey{/tcb/poster/prefix} for
  the embedded poster or you surely get a total mess.
  \end{marker}

  There are several properties inside a \refEnv{tcbposter} which may be useful
  for advanced code (skip the following on first reading):
  \begin{itemize}
  \item\docAuxCommand{tcbposterwidth}: Width of the poster (without margins).
  \item\docAuxCommand{tcbposterheight}: Height of the poster (without margins).
  \item\docAuxCommand{tcbpostercolspacing}: Column distance.
  \item\docAuxCommand{tcbposterrowspacing}: Row distance.
  \item\docAuxCommand{tcbpostercolumns}: Column quantity.
  \item\docAuxCommand{tcbposterrows}: Row quantity.
  \item\docAuxCommand{tcbpostercolwidth}: Width of a column.
  \item\docAuxCommand{tcbposterrowheight}: Height of a row.
  \end{itemize}

\medskip
\begin{docCommand}[doc new=2017-07-03]{tcbposterset}{\marg{options}}
  Sets options for every following \refEnv{tcbposter} inside the current \TeX\ group.
  For example, the numbers for rows and columns may be defined for the whole document by this:
\begin{dispListing}
\tcbposterset{poster={columns=2,rows=3}}
\end{dispListing}
  See \refEnv{tcbposter} for all feasible options.
\end{docCommand}


\clearpage
%--------------------------
\subsection{Poster Settings}\label{subsec:poster_settings}

\begin{postersetTcbKey}[][doc new=2017-07-03]{poster}{=\marg{option list}}{style, no default}
  This option can be applied inside \refEnv{tcbposter} and \refCom{tcbposterset}
  to set the given poster \meta{option list}, e.g.
\begin{dispListing}
\tcbposterset{poster={width=20cm,height=15cm}}
\end{dispListing}
  For the \meta{option list}, see the following keys.
\end{postersetTcbKey}


\begin{posterTcbKey}[][doc new=2017-07-03]{columns}{=\meta{number}}{no default, initially |3|}
  Sets the \meta{number} of columns for a |tcbposter|.
\begin{exdispExample}{columns}
  \begin{tcbposter}[
    poster = {showframe,columns=5,rows=2,spacing=1mm,height=4cm},
  ]
  \end{tcbposter}
\end{exdispExample}
\end{posterTcbKey}

\begin{posterTcbKey}[][doc new=2017-07-03]{rows}{=\meta{number}}{no default, initially |4|}
  Sets the \meta{number} of rows for a |tcbposter|.
\end{posterTcbKey}


\begin{posterTcbKey}[][doc new=2017-07-03]{colspacing}{=\meta{length}}{no default, initially |4mm|}
  Sets \meta{length} as distance between columns.
\end{posterTcbKey}

\begin{posterTcbKey}[][doc new=2017-07-03]{rowspacing}{=\meta{length}}{no default, initially |4mm|}
  Sets \meta{length} as distance between rows.
\end{posterTcbKey}

\begin{posterTcbKey}[][doc new=2017-07-03]{spacing}{=\meta{length}}{style, no default, initially |4mm|}
  Sets \meta{length} as distance between columns and rows.
\end{posterTcbKey}


\begin{posterTcbKey}[][doc new=2017-07-03]{showframe}{\colOpt{=true\textbar false}}{default |true|, initially |false|}
  Displays a red auxiliary mesh as optical support during poster creation.
  Also, every \refKey{/tcb/posterloc/name} is displayed.
\end{posterTcbKey}


\begin{posterTcbKey}[][doc new=2017-07-03]{width}{=\meta{length}}{no default, initially \cs{linewidth}}
  Sets \meta{length} as width of the poster. For a typical poster, this has not
  to be set manually.
\end{posterTcbKey}

\begin{posterTcbKey}[][doc new=2017-07-03]{height}{=\meta{length}}{no default, initially unset}
  Sets \meta{length} as height of the poster. For a typical poster, this has not
  to be set manually, but is set automatically to an appropriate value.
\end{posterTcbKey}


\begin{posterTcbKey}[][doc new=2017-07-03]{prefix}{=\meta{name}}{no default, initially |TCBPOSTER@|}
  \meta{name} is set as prefix for any \tikzname\ node which is generated
  automatically by the \mylib{poster} library. This encompasses predefined
  nodes like |top|, |bottom|, \ldots, and nodes defined by using
  \refKey{/tcb/posterloc/name}. Also, see~\Fullref{subsec:poster_environment}.
  For a typical poster, this value can stay as it is.
\end{posterTcbKey}


%--------------------------
\subsection{Coverage}\label{subsec:poster_coverage}

\begin{postersetTcbKey}[][doc new=2017-07-03]{coverage}{=\marg{option list}}{style, no default}
  This option can be applied inside \refEnv{tcbposter} and \refCom{tcbposterset}
  and it adds an optional coverage for the poster which is a surrounding |tcolorbox|
  with the given \meta{option list}. Here, margins and background settings
  for the poster can be given.
  The \emph{coverage} has several default |tcolorbox| settings
  suitable for the purpose:
\begin{dispListing}
enhanced, frame hidden, sharp corners, boxsep=0pt, boxrule=0pt,
top=4mm, bottom=4mm, left=4mm, right=4mm,
toptitle=2mm, bottomtitle=2mm, colback=white
\end{dispListing}

The \meta{option list} can contain any |tcolorbox| option.

\begin{exdispExample}{coverage}
\begin{tcbposter}[
  poster   = {showframe,spacing=1mm},
  coverage = {height=5cm,
              interior style={top color=yellow,bottom color=yellow!50!red},
              watermark text={My Poster},watermark color=white,
             },
]
\end{tcbposter}
\end{exdispExample}

\begin{itemize}
\item For a typical poster, the option \refKey{/tcb/spread} will use the
  whole page for the poster coverage.
\item Poster margins can be adapted by \refKey{/tcb/left}, \refKey{/tcb/right},
  \refKey{/tcb/top}, \refKey{/tcb/bottom}.
\item Poster background can be changed by \refKey{/tcb/colback},
  \refKey{/tcb/interior style}, \refKey{/tcb/interior style image}, etc.
\end{itemize}
\end{postersetTcbKey}


\begin{postersetTcbKey}[][doc new=2017-07-03]{no coverage}{}{style, no value, initially set}
  Removes the surrounding |tcolorbox| completely.
\end{postersetTcbKey}

\clearpage
%--------------------------
\subsection{Common Box Settings}\label{subsec:poster_boxsettings}


\begin{postersetTcbKey}[][doc new=2017-07-03]{boxes}{=\marg{option list}}{style, no default}
  This option can be applied inside \refEnv{tcbposter} and \refCom{tcbposterset}
  and it is used to set up the style of the |tcolorbox|es inside the poster.
  The \meta{option list} can contain any |tcolorbox| option, but box size
  options are not assumed to be useful here, because the size will be
  determined by the placement options.

\begin{exdispExample}{boxes}
\begin{tcbposter}[
  poster   = {spacing=2mm,columns=3,rows=2},
  coverage = {height=5cm,
              interior style={top color=yellow,bottom color=yellow!50!red},
             },
  boxes    = {sharp corners=downhill,arc=3mm,boxrule=1mm,
              colback=white,colframe=cyan,
              title style={left color=black,right color=cyan},
              fonttitle=\bfseries\scshape}
]
  \posterbox[adjusted title=First]{column=1,row=1,span=2}{First box}
  \posterbox[adjusted title=Second]{column=1,row=2,span=2}{Second box}
  \posterbox[adjusted title=Third]{column=3,row=1,rowspan=2}{Third box}
\end{tcbposter}
\end{exdispExample}

\end{postersetTcbKey}


%--------------------------
\subsection{Font Scaling}\label{subsec:poster_fontsize}

\begin{postersetTcbKey}[][doc new=2017-07-03]{fontsize}{=\meta{length}}{style, no default, initially unset}
  This option can be applied inside \refEnv{tcbposter} and \refCom{tcbposterset}.
  It uses \refKey{/tcb/fit basedim} and \refKey{/tcb/fit fontsize macros}
  to redefine |\normalsize| to \meta{length} and all other standard
  font size macros like |\small| and |\large| accordingly.\par
  This needs a freely scalable font family like |lmodern| to work.
  If \refKey{/tcb/posterset/fontsize} is not applied, there standard
  font size macros are not changed in any way.

\begin{dispListing}
\begin{tcbposter}[
  poster   = {spacing=2mm,columns=3,rows=2},
  coverage = {height=5cm,
              interior style={top color=yellow,bottom color=yellow!50!red},
             },
  fontsize = 15pt,   % <--- \normalsize is now 15pt
]
...
\end{dispListing}
\end{postersetTcbKey}


\clearpage
%--------------------------
\subsection{Box Placement}\label{subsec:poster_boxplacement}

\begin{docCommand}[doc new=2017-07-03]{posterbox}{\oarg{options}\marg{placement}\marg{box content}}
  Inside a \refEnv{tcbposter} environment, this places a |tcolorbox| with
  additional |tcolorbox| \meta{options} and the given \meta{box content}
  at a place determined by \meta{placement}.
  All \meta{placement} options are described in the following.
  Note that \meta{box content} cannot contain \emph{verbatim} material,
  see \refEnv{posterboxenv}.
\begin{exdispExample}{posterbox}
\begin{tcbposter}[
  poster = {showframe,height=4cm,spacing=2mm,rows=2},
  boxes  = {beamer,colframe=blue!50!black,colback=blue!50,colupper=yellow!50},
]
\posterbox[title=My title]{name=A,column=2,row=2}{My first box}
\end{tcbposter}
\end{exdispExample}
\end{docCommand}

\begin{docEnvironment}[doc new=2017-07-03]{posterboxenv}{\oarg{options}\marg{placement}}
  This is the environment version of \refCom{posterbox}, i.e.\ inside a
  \refEnv{tcbposter} environment, this places a |tcolorbox| with
  additional |tcolorbox| \meta{options} and the given \meta{environment content}
  at a place determined by \meta{placement}.
  In contrast to \refCom{posterbox}, the \meta{environment content} is
  allowed to contain \emph{verbatim} material. Note that the implementation
  of \refCom{posterbox} is more efficient than the implementation of \refEnv{posterboxenv}.

\enlargethispage*{1cm}
\begin{exdispExample}{posterboxenv}
\begin{tcbposter}[
  poster = {showframe,height=4cm,spacing=2mm,rows=2},
  boxes  = {size=small,beamer,
            colframe=blue!50!black,colback=blue!50,colupper=yellow!50},
]
\begin{posterboxenv}[title=My title]{name=A,column=2,between=top and bottom}
  My first box.
  \begin{tcblisting}{size=small,colback=yellow!10}
My \textbf{first}
poster listing.
  \end{tcblisting}
\end{posterboxenv}
\end{tcbposter}
\end{exdispExample}

\end{docEnvironment}


\clearpage
\begin{posterlocTcbKey}[][doc new=2017-07-03]{name}{=\meta{name}}{no default, initially |@|}
  Sets \meta{name} as reference for the current \refCom{posterbox} or
  \refEnv{posterboxenv}.
  A \tikzname\ shape name is constructed automatically as combination
  of \refKey{/tcb/poster/prefix} and \meta{name}.
\begin{exdispExample}{name}
\begin{tcbposter}[
  poster = {showframe,height=2.5cm,spacing=2mm,rows=2},
  boxes  = {beamer,colframe=blue!50!black,colback=blue!50,colupper=yellow!50},
]
\posterbox{name=A,column=2,row=2}{My first box}
\node[below right=4mm,fill=yellow] (X) at (TCBPOSTER@poster.north west) {Example A};
\draw[blue,very thick,->] (X) |- (TCBPOSTER@A);
\end{tcbposter}
\end{exdispExample}
\end{posterlocTcbKey}


\begin{posterlocTcbKey}[][doc new=2017-07-03]{column}{=\meta{number}}{no default, initially |1|}
  Places the box at the column denoted by \meta{number}. If \refKey{/tcb/posterloc/span}
  is not |1|, the box is aligned to the left side of column \meta{number}.
\begin{exdispExample}{column}
\begin{tcbposter}[
  poster = {showframe,height=2.5cm,spacing=2mm,rows=2},
  boxes  = {beamer,colframe=blue!50!black,colback=blue!50,colupper=yellow!50},
]
\posterbox{row=1,column=2,span=2}{First box}
\posterbox{row=2,column=2,span=0.8}{Second box}
\end{tcbposter}
\end{exdispExample}
\end{posterlocTcbKey}

\enlargethispage*{1cm}
\begin{posterlocTcbKey}[][doc new=2017-07-03]{column*}{=\meta{number}}{no default, initially unset}
  Places the box at the column denoted by \meta{number}. If \refKey{/tcb/posterloc/span}
  is not |1|, the box is aligned to the right side of column \meta{number}.
\begin{exdispExample}{columnstar}
\begin{tcbposter}[
  poster = {showframe,height=2.5cm,spacing=2mm,rows=2},
  boxes  = {beamer,colframe=blue!50!black,colback=blue!50,colupper=yellow!50},
]
\posterbox{row=1,column*=2,span=2}{First box}
\posterbox{row=2,column*=2,span=0.8}{Second box}
\end{tcbposter}
\end{exdispExample}
\end{posterlocTcbKey}


\clearpage
\begin{posterlocTcbKey}[][doc new=2017-07-03]{span}{=\meta{number}}{no default, initially |1|}
  Sets the width of the current box to span \meta{number} columns.
  \meta{number} is also allowed to be a real number like |0.5| or |1.7|.
  See \refKey{/tcb/posterloc/column} and \refKey{/tcb/posterloc/column*}
  for examples.
\end{posterlocTcbKey}

\begin{posterlocTcbKey}[][doc new=2017-07-03]{row}{=\meta{number}}{no default, initially unset}
  If this option is applied, the box is placed at the row denoted by \meta{number}.
  Also, the height is set as fixed according to \refKey{/tcb/posterloc/rowspan}.
\begin{exdispExample}{row}
\begin{tcbposter}[
  poster = {showframe,height=2.5cm,spacing=2mm,rows=2},
  boxes  = {beamer,colframe=blue!50!black,colback=blue!50,colupper=yellow!50},
]
\posterbox{row=1,column=1}{First box}
\posterbox{row=1,column=2,rowspan=2}{Second box}
\posterbox[natural height]{row=1,column=3}{Third box}
\end{tcbposter}
\end{exdispExample}
\end{posterlocTcbKey}

\begin{posterlocTcbKey}[][doc new=2017-07-03]{rowspan}{=\meta{number}}{no default, initially |1|}
  Sets the height of the current box to span \meta{number} rows.
  \meta{number} is also allowed to be a real number like |0.5| or |1.7|.
\begin{exdispExample}{rowspan}
\begin{tcbposter}[
  poster = {showframe,height=2.5cm,spacing=2mm,rows=2},
  boxes  = {beamer,colframe=blue!50!black,colback=blue!50,colupper=yellow!50},
]
\posterbox{row=1,column=1,rowspan=0.9}{First box}
\posterbox{row=1,column=2,rowspan=1.5}{Second box}
\posterbox{row=1,column=3,rowspan=2}{Third box}
\end{tcbposter}
\end{exdispExample}
\end{posterlocTcbKey}

\begin{posterlocTcbKey}[][doc new=2017-07-03]{fixed height}{}{no value, initially |0pt|}
  Sets the height of the current box span rows as denoted by
  \refKey{/tcb/posterloc/rowspan}.
  This can be used, if not \refKey{/tcb/posterloc/row}, but another
  height placement option is applied.
\end{posterlocTcbKey}


\clearpage
\begin{posterlocTcbKey}[][doc new=2017-07-03]{below}{=\meta{name}}{no default, initially |top|}
  The box is placed below another box with the given \meta{name}. Also,
  \meta{name} can be a predefined node, see \Fullref{subsec:poster_environment}.
\begin{exdispExample}{below}
\begin{tcbposter}[
  poster = {showframe,height=3cm,spacing=2mm,rows=2},
  boxes  = {beamer,colframe=blue!50!black,colback=blue!50,colupper=yellow!50},
]
\posterbox{name=A,column=1,below=top}{First box}
\posterbox{name=B,column=1,below=A}{Second box}
\posterbox{name=C,column=2,below=B}{Third box}
\posterbox{name=D,column=3,below=row1}{Fourth box}
\end{tcbposter}
\end{exdispExample}
\end{posterlocTcbKey}


\begin{posterlocTcbKey}[][doc new=2017-07-03]{above}{=\meta{name}}{no default, initially unset}
  The box is placed above another box with the given \meta{name}. Also,
  \meta{name} can be a predefined node, see \Fullref{subsec:poster_environment}.
\begin{exdispExample}{above}
\begin{tcbposter}[
  poster = {showframe,height=3cm,spacing=2mm,rows=2},
  boxes  = {beamer,colframe=blue!50!black,colback=blue!50,colupper=yellow!50},
]
\posterbox{name=A,column=1,above=bottom}{First box}
\posterbox{name=B,column=1,above=A}{Second box}
\posterbox{name=C,column=2,above=B}{Third box}
\posterbox{name=D,column=3,above=row2}{Fourth box}
\end{tcbposter}
\end{exdispExample}
\end{posterlocTcbKey}


\clearpage
\begin{posterlocTcbKey}[][doc new=2017-07-03]{at}{=\meta{name}}{no default, initially unset}
  The box is placed at the position with the given \meta{name}. This is
  quite likely a predefined node, see \Fullref{subsec:poster_environment}.
\begin{exdispExample}{at}
\begin{tcbposter}[
  poster = {showframe,height=3cm,spacing=2mm,rows=2},
  boxes  = {beamer,colframe=blue!50!black,colback=blue!50,colupper=yellow!50},
]
\posterbox{name=A,column=1,at=middle}{First box}
\posterbox{name=B,column=2,at=row1}{Second box}
\end{tcbposter}
\end{exdispExample}
\end{posterlocTcbKey}


\begin{posterlocTcbKey}[][doc new=2017-07-03]{between}{=\meta{name1} and \meta{name2}}{no default, initially unset}
  The box is placed below a box \meta{name1} and above another box \meta{name2}. Also,
  \meta{name1} and \meta{name2} can be predefined nodes, see \Fullref{subsec:poster_environment}.
\begin{exdispExample}{between}
\begin{tcbposter}[
  poster = {showframe,height=3cm,spacing=2mm,rows=2},
  boxes  = {beamer,colframe=blue!50!black,colback=blue!50,colupper=yellow!50},
]
\posterbox{name=A,column=1,below=top}{First box}
\posterbox{name=B,column=1,between=A and bottom}{Second box}
\posterbox{name=C,column=2,above=bottom}{Third box}
\posterbox{name=D,column=2,between=top and C,span=2}{Fourth box}
\posterbox{name=E,column=3,between=D and bottom}{Fifth box}
\end{tcbposter}
\end{exdispExample}
\end{posterlocTcbKey}


\clearpage
\begin{posterlocTcbKey}[][doc new=2017-07-03]{sequence}{=\meta{sequence}}{no default, initially unset}
  The box is broken into partial boxes. These partial boxes are placed
  following the given \meta{sequence} of placements.
  The feasible syntax for the \meta{sequence} is:\par\medskip
  \meta{column a} |between| \meta{name a1} |and| \meta{name a2} |then|\\
  \meta{column b} |between| \meta{name b1} |and| \meta{name b2} |then|\\
  \meta{column c} |between| \meta{name c1} |and| \meta{name c2} |then|\ldots\par\medskip
  Obviously, this places the first part box at \meta{column a} between
  \meta{name a2} and \meta{name a2}. The second box part is placed
  at \meta{column b} between
  \meta{name b2} and \meta{name b2}, and so on.

\begin{exdispExample}{sequence}
\begin{tcbposter}[
  poster = {showframe,height=6cm,spacing=2mm,rows=2},
  boxes  = {beamer,colframe=blue!50!black,colback=blue!50,colupper=yellow!50},
]
\posterbox[adjusted title=A]{name=A,column=1,below=top,span=2}{First box}
\posterbox{name=B,column=2,above=bottom,span=2}{Second box}
\posterbox[adjusted title=C,colframe=red!50!black,colback=red!50]{
  name=C, sequence=1 between A and bottom then
                   2 between A and B then
                   3 between top and B
  }{\lipsum[2]}
\end{tcbposter}
\end{exdispExample}
\end{posterlocTcbKey}

\enlargethispage*{2cm}
\begin{docTcbKey}[][doc new=2017-07-03]{placeholder}{}{style, no value}
  If the box content of a \refKey{/tcb/posterloc/sequence} is too short
  to fill all reserved box parts, the empty boxes are drawn with
  the \refKey{/tcb/placeholder} style. This style can be redefined, e.g.
  to \refKey{/tcb/blankest}, if nothing should be drawn for empty boxes.
\begin{exdispExample}{placeholder}
\begin{tcbposter}[
  poster = {showframe,height=2.5cm,spacing=2mm,rows=2},
  boxes  = {beamer,colframe=blue!50!black,colback=blue!50,colupper=yellow!50},
]
\posterbox{name=A,column=1,below=top,span=2}{First box}
\posterbox[colframe=red!50!black,colback=red!50]{
  name=B, sequence=1 between A and bottom then
                   2 between A and bottom then
                   3 between top and bottom
  }{Second box followed by placeholder boxes}
\end{tcbposter}
\end{exdispExample}
\end{docTcbKey}


\clearpage
\begin{posterlocTcbKey}[][doc new=2017-07-03]{xshift}{=\meta{length}}{no default, initially |0pt|}
  Horizontal shift of a box by \meta{length}.
\begin{exdispExample}{xshift}
\begin{tcbposter}[
  poster = {showframe,height=3cm,spacing=2mm,rows=2},
  boxes  = {beamer,colframe=blue!50!black,colback=blue!50,colupper=yellow!50},
]
\posterbox{name=A,column=1,row=1,xshift=6mm}{First box}
\posterbox{name=B,column=2,row=2,xshift=-6mm}{Second box}
\end{tcbposter}
\end{exdispExample}
\end{posterlocTcbKey}


\begin{posterlocTcbKey}[][doc new=2017-07-03]{yshift}{=\meta{length}}{no default, initially |0pt|}
  Vertical shift of a box by \meta{length}.
\begin{exdispExample}{yshift}
\begin{tcbposter}[
  poster = {showframe,height=3cm,spacing=2mm,rows=2},
  boxes  = {beamer,colframe=blue!50!black,colback=blue!50,colupper=yellow!50},
]
\posterbox{name=A,column=1,row=1,yshift=-4mm}{First box}
\posterbox{name=B,column=2,row=2,yshift=4mm}{Second box}
\end{tcbposter}
\end{exdispExample}
\end{posterlocTcbKey}

