%
\documentclass[final,dvipsnames]{beamer}
\usepackage[scale=0.90]{beamerposter} %scale is for fontsize?
\usepackage[absolute,overlay]{textpos}
\usepackage{color}
\usepackage{tikz}
%\usepackage{calc}
\usetikzlibrary{shadings}
\usepackage{amsmath,amssymb,latexsym}
\usepackage[many]{tcolorbox}

%\usetheme{confposter}

\newtcolorbox{myblock}[1][]{
  beamer,
  width=\textwidth+7pt,
  enlarge left by=-3pt,
  colframe=block body.bg,
  bottom=0pt,
  top=-2pt,
  left=0pt,
  right=0pt,
  toptitle=-1pt,
  bottomtitle=-1pt,
  fonttitle=\normalfont,
  adjusted title=#1,
  interior titled code={
    \shade[left color=Maroon!80,right color=Dandelion,middle color=Salmon] 
      (title.south west) --
      (title.south east) {[rounded corners] -- 
      (title.north east)  -- 
      (title.north west)} --
      (title.south west); 
  }
}
\title{Title}
\author{Some People}
\institute{Department of Blah}

\begin{document}
\newsavebox\mybox
\setbox0=\vtop{%
		\begin{block}{A standard block} 
			some text  
		\end{block}%
}
\newlength{\bh}
\setlength{\bh}{\ht0}%
\addtolength{\bh}{\dp0}%
\begin{frame}
\begin{columns}
\column{0.3\textwidth}
\the\bh
\usebox0 %prints the box visibly
\usebox\mybox % does not print it visibly
%\begin{block}{A standard block}
%This box ia a box provided by the \texttt{beamer} class.
%\end{block}
%
%\begin{myblock}[An example with \texttt{tcolorbox}]
%This box looks like a box provided by the \texttt{beamer} class.
%\end{myblock}
\end{columns}
\end{frame}

\end{document}

